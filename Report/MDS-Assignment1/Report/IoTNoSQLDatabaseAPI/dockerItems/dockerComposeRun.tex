\paragraph{Run docker compose files}\mbox{}\\
setup replica $\xrightarrow{}$ set nodes - 5 nodes per 3 replica sets $\xrightarrow{}$ 15 docker nodes are created
\begin{lstlisting}[language=Bash, caption=Create 15 Mongo nodes]
sudo docker-compose -f docker-compose.yaml up -d
\end{lstlisting}
setup config $\xrightarrow{}$ 3 config (docker) servers are running
\begin{lstlisting}[language=Bash, caption=Create config servers]
sudo docker-compose -f mongod.yaml up -d
\end{lstlisting}
setup router $\xrightarrow{}$ 3 mongos (router) dockers are running
\begin{lstlisting}[language=Bash, caption=Create router servers]
sudo docker-compose -f mongos.yaml up -d
\end{lstlisting}

%=========================================================================================================
% configure config servers replica set
%=========================================================================================================
\paragraph{Configure config servers replica set}\mbox{}\\
Replica set 1
\begin{lstlisting}[language=Bash, caption=Set config1]
docker exec -it mongo_Config1 bash -c "echo 'rs.initiate(
    {_id: \"mongo_ReplicaSet1_Conf\",configsvr: true, members: [
        { _id : 0, host : \"mongo_Config1\" },
        { _id : 1, host : \"mongo_Config2\" }, 
        { _id : 2, host : \"mongo_Config3\" }
        ]
    })
' | mongosh"
\end{lstlisting}
Replica set 2
\begin{lstlisting}[language=Bash, caption=Set config2]
docker exec -it mongo_Config2 bash -c "echo 'rs.initiate(
    {_id: \"mongo_ReplicaSet1_Conf\",configsvr: true, members: [
        { _id : 0, host : \"mongo_Config1\" },
        { _id : 1, host : \"mongo_Config2\" }, 
        { _id : 2, host : \"mongo_Config3\" }
        ]
    }
)' | mongosh"
\end{lstlisting}
Replica set 3
\begin{lstlisting}[language=Bash, caption=Set config3]
docker exec -it mongo_Config3 bash -c "echo 'rs.initiate(
    {_id: \"mongo_ReplicaSet1_Conf\",configsvr: true, members: [
        { _id : 0, host : \"mongo_Config1\" },
        { _id : 1, host : \"mongo_Config2\" }, 
        { _id : 2, host : \"mongo_Config3\" }
        ]
    }
)' | mongosh"
\end{lstlisting}

%=========================================================================================================
% check config server replica set status
%=========================================================================================================
\paragraph{Check config server replica set status}\mbox{}\\
\begin{lstlisting}[language=Bash, caption=Check replica set status]
docker exec -it mongo_Config1 bash -c "echo 'rs.status()' | mongosh"
\end{lstlisting}

%=========================================================================================================
% build shard replica set
%=========================================================================================================
\paragraph{Build shard replica set}\mbox{}\\
Replica set 1
\begin{lstlisting}[language=Bash, caption=Build replica set 1]
docker exec -it mongo_ReplicaSet1_Node1 bash -c "echo 'rs.initiate(
    {_id : \"mongo_ReplicaSet1\", members: [
        { _id : 0, host : \"mongo_ReplicaSet1_Node1\" },
        { _id : 1, host : \"mongo_ReplicaSet1_Node2\" },
        { _id : 2, host : \"mongo_ReplicaSet1_Node3\" },
        { _id : 3, host : \"mongo_ReplicaSet1_Arbiter\", arbiterOnly: true},
        { _id : 4, host : \"mongo_ReplicaSet1_Backup\", hidden: true}
        ]
    }
)' | mongosh"
\end{lstlisting}
Replica set 2
\begin{lstlisting}[language=Bash, caption=Build replica set 2]
docker exec -it mongo_ReplicaSet2_Node1 bash -c "echo 'rs.initiate(
    {_id : \"mongo_ReplicaSet2\", members: [
        { _id : 0, host : \"mongo_ReplicaSet2_Node1\" },
        { _id : 1, host : \"mongo_ReplicaSet2_Node2\" },
        { _id : 2, host : \"mongo_ReplicaSet2_Node3\" },
        { _id : 3, host : \"mongo_ReplicaSet2_Arbiter\", arbiterOnly: true},
        { _id : 4, host : \"mongo_ReplicaSet2_Backup\", hidden: true}
        ]
    }
)' | mongosh"
\end{lstlisting}
Replica set 3
\begin{lstlisting}[language=Bash, caption=Build replica set 3]
docker exec -it mongo_ReplicaSet3_Node1 bash -c "echo 'rs.initiate(
    {_id : \"mongo_ReplicaSet3\", members: [
        { _id : 0, host : \"mongo_ReplicaSet3_Node1\" },
        { _id : 1, host : \"mongo_ReplicaSet3_Node2\" },
        { _id : 2, host : \"mongo_ReplicaSet3_Node3\" },
        { _id : 3, host : \"mongo_ReplicaSet3_Arbiter\", arbiterOnly: true},
        { _id : 4, host : \"mongo_ReplicaSet3_Backup\", hidden: true}
        ]
    }
)' | mongosh"
\end{lstlisting}

%=========================================================================================================
% check status primary-secondary
%=========================================================================================================
\paragraph{Check status primary-secondary}\mbox{}\\
Status for replica set 1
\begin{lstlisting}[language=Bash, caption=Check status of replica set 1]
docker exec -it mongo_ReplicaSet1_Node1 bash -c "echo 'rs.status()' | mongosh"
\end{lstlisting}
Status for replica set 2
\begin{lstlisting}[language=Bash, caption=Check status of replica set 2]
docker exec -it mongo_ReplicaSet2_Node1 bash -c "echo 'rs.status()' | mongosh"
\end{lstlisting}
Status for replica set 3
\begin{lstlisting}[language=Bash, caption=Check status of replica set 3]
docker exec -it mongo_ReplicaSet3_Node1 bash -c "echo 'rs.status()' | mongosh"
\end{lstlisting}

%=========================================================================================================
% introduce shard to the routers
%=========================================================================================================
\paragraph{Introduce replica set to the routers}\mbox{}\\
Add mongo{\_}Shard1
\begin{lstlisting}[language=Bash, caption=Introduce replica 1]
docker exec -it mongo_Shard1 bash -c "echo 'sh.addShard(\"mongo_ReplicaSet1/mongo_ReplicaSet1_Node1\")' | mongosh"
\end{lstlisting}
Add mongo{\_}Shard2
\begin{lstlisting}[language=Bash, caption=Introduce replica 2]
docker exec -it mongo_Shard2 bash -c "echo 'sh.addShard(\"mongo_ReplicaSet2/mongo_ReplicaSet2_Node1\")' | mongosh"
\end{lstlisting}
Add mongo{\_}Shard3
\begin{lstlisting}[language=Bash, caption=Introduce replica 3]
docker exec -it mongo_Shard3 bash -c "echo 'sh.addShard(\"mongo_ReplicaSet3/mongo_ReplicaSet3_Node1\")' | mongosh"
\end{lstlisting}

%=========================================================================================================
% get shard status
%=========================================================================================================
\paragraph{Get shard status}\mbox{}\\
\begin{lstlisting}[language=Bash, caption=Get shard status]
docker exec -it mongo_Shard1 bash -c "echo 'sh.status()' | mongosh"
\end{lstlisting}

%=========================================================================================================
% create test db
%=========================================================================================================
\paragraph{Create test db}\mbox{}\\
Create test db in shard mongo{\_}ReplicaSet{\_}1 for testing.
\begin{lstlisting}[language=Bash, caption=Create test db]
docker exec -it mongo_ReplicaSet1_Node1 bash -c "echo 'use testDb' | mongosh"
\end{lstlisting}

%=========================================================================================================
% enable sharding for new db
%=========================================================================================================
\paragraph{Enable sharding for new db}\mbox{}\\
Test sharding of new database.
\begin{lstlisting}[language=Bash, caption=Enable sharding for new db]
docker exec -it mongo_Shard1 bash -c "echo 'sh.enableSharding(\"testDb\")' | mongosh"
\end{lstlisting}

%=========================================================================================================
% create test collection
%=========================================================================================================
\paragraph{Create test collection}\mbox{}\\
Create test collection for further testing.
\begin{lstlisting}[language=Bash, caption=Create test collection RS1]
docker exec -it mongo_ReplicaSet1_Node1 bash -c "echo 'db.createCollection(\"testDb.testCollection\")' | mongosh"
\end{lstlisting}
\begin{lstlisting}[language=Bash, caption=Create test collection RS2]
docker exec -it mongo_ReplicaSet2_Node1 bash -c "echo 'db.createCollection(\"testDb.testCollection\")' | mongosh"
\end{lstlisting}
\begin{lstlisting}[language=Bash, caption=Create test collection RS3]
docker exec -it mongo_ReplicaSet3_Node1 bash -c "echo 'db.createCollection(\"testDb.testCollection\")' | mongosh"
\end{lstlisting}

%=========================================================================================================
% shard based on the key
%=========================================================================================================
\paragraph{Shard based on the key}\mbox{}\\
Test sharding collection based on the key.
\begin{lstlisting}[language=Bash, caption=Shard based on the key]
docker exec -it mongo_Shard1 bash -c "echo 'sh.shardCollection(\"testDb.testCollection\", {\"shardingField\" : 1})' | mongosh"
\end{lstlisting}